\documentclass[letterpaper,10pt]{article} % Default font size and paper size

\usepackage{fontspec} % For loading fonts
\defaultfontfeatures{Mapping=tex-text}
\setmainfont{Droid Sans} % Main document font

\usepackage{xunicode,xltxtra,url,parskip} % Formatting packages

\usepackage[usenames,dvipsnames]{xcolor} % Required for specifying custom colors

\usepackage[margin=0.5in]{geometry}
% To reduce the height of the top margin uncomment: \addtolength{\voffset}{-1.3cm}

\usepackage{hyperref} % Required for adding links	and customizing them
\definecolor{linkcolour}{rgb}{0,0.2,0.6} % Link color
\hypersetup{colorlinks,breaklinks,urlcolor=linkcolour,linkcolor=linkcolour} % Set link colors throughout the document

\usepackage{titlesec} % Used to customize the \section command
\titleformat{\section}{\Large\scshape\raggedright}{}{0em}{}[\titlerule] % Text formatting of sections
\titlespacing{\section}{0pt}{3pt}{3pt} % Spacing around sections
\titlespacing{\subsection}{0pt}{10pt}{*0}
\titlespacing{\paragraph}{0pt}{*0}{10pt}

\usepackage{longtable}

\begin{document}

\pagestyle{empty} % Removes page numbering
\urlstyle{same}

\font\fb=''[cmr10]'' % Change the font of the \LaTeX command under the skills section

\par{\centering{\Huge Ten-Young Guh}\par} % Your name
\par{\centering{
    67-62 Exeter St, Forest Hills, NY, 11375
    | \href{mailto:tenyoung795@gmail.com}{tenyoung795@gmail.com}
    | \href{tel:+1-347-756-8608}{+1-347-756-8608}
}\par}
\par{\centering{
    LinkedIn: \url{https://www.linkedin.com/profile/view?id=201881408}
}\par}
\par{\centering{
    Github: \url{http://github.com/tenyoung795}
}\par}

\section{Education}

\begin{longtable}{r|p{4.25in}}	
    September 2012-May 2016 (expected) & \textbf{Binghamton University, State University of New York} \\
    & \textbf{The Thomas J. Watson School of Engineering} \\
    & \textbf{Bachelor of Science in Engineering} \\
    & 3.952/4.000 GPA, Dean's List \\
    & Relevant coursework: Computer Systems 1 to 3, Data Structures and Algorithms,
    Ernst \& Young Trajectory Program {\footnotesize (a joint management-computer science program)} \\
\end{longtable}

\section{Technical Skills}

\paragraph{Programming Languages} C/C++(14), Java, Python, Objective-C, {\footnotesize bash, JavaScript, ActionScript, Go, Haskell}
\paragraph{Platforms} Unix, iOS, Android
\paragraph{Frameworks} Core Data, OCMock, Keep-It-Functional (KIF), JUnit, Spring
\paragraph{Editors} IntelliJ, Eclipse, Vim {\footnotesize (preference of IntelliJ over Eclipse)}
\paragraph{Version control} Git, Perforce

Minimal skill in CSS and HTML

\section{Internship Experience}

\begin{longtable}{r|p{4.5in}}
    June 2014-August 2014 & \textbf{Software-Engineer-in-Test Intern at Google} \\
    New York, NY & Programmed version 0 of an iOS app \\
    & \begin{itemize}
        \item Implemented user interface using Google's Material design
        \item Persisted data onto the device using Core Data, Apple's standard data persistence framework
        \item Tested user interface using KIF
        \item Managed code using Git wrapper around Perforce
    \end{itemize} \\

    \multicolumn{2}{c}{} \\

    June 2013-September 2013 & \textbf{Software Engineering Intern at Energyhub} \\
    Brooklyn, NY & \begin{itemize}
        \item Programmed a proxy server for Carrier's thermostat API in Go
        \item Implemented an endpoint for Carrier's theromstat API
        \item
            Implemented a Southern California Edison advertisement from a Spring thermostat server
            to an Android thermostat app
        \item
            Integrated Carrier’s thermostat project programmed in Flex into a Jenkins environment;
            implemented the project’s first unit test
    \end{itemize} \\

    \multicolumn{2}{c}{} \\

    January 2013 & \textbf{Programming Intern at Halo Neuro} \\
    New York, NY & \begin{itemize}
        \item Began programming the iPhone app for the Halo headband
        \item Programmed a reaction time game in Flash called React
    \end{itemize} \\

    \multicolumn{2}{c}{} \\

    \newpage

    July 2012-August 2012 & \textbf{Engineering Intern at Peek} \\
    New York, NY & \begin{itemize}
        \item Programmed an Android game in which you control a taxi driver in Delhi
        \item Programmed the client-side of an app store in Lua for a MediaTek phone
    \end{itemize}
\end{longtable}

\section{Personal Team Projects}

\begin{longtable}{r|p{4.5in}}
    April 2014-Present & \textbf{Crowdshop Android App} \\
    Started at Bitcamp & \textbf{Github: \url{https://github.com/leeeonlee/crowdshop-app}} \\
    Hackathon at UMD & Android app to pay friends to shop for you with an optional reward \\
    & \begin{itemize}
        \item Implemented most of user interface
        \item Handled HTTP requests in the background using RoboSpice
    \end{itemize} \\

    \multicolumn{2}{c}{} \\

    April 2014 & \textbf{Schedulizer} \\
    Started at HackBU & \textbf{Github: \url{https://github.com/leeeonlee/fsched}} \\
    Hackathon at SUNY Binghamton & Web app to assist Binghamton students plan their schedule \\
    & Implemented course reference-matching algorithm \\ 
    & \begin{itemize}
        \item Matched a course's references by section
        \item Matched lectures with the correct activity
    \end{itemize} \\

    \multicolumn{2}{c}{} \\

    April 2014 & \textbf{Rotapong} \\
    Started at PennApps  & \textbf{Github: \url{https://github.com/tenyoung795/Rotapong}} \\
    Hackathon at UPenn & Android version of Pong using device rotation to control the paddle \\
    & \begin{itemize}
        \item Drew objects using OpenGL
        \item Implemented a peer-to-peer protocol using UDP to support multiplayer
    \end{itemize}

\end{longtable}

\section{Educational Individual Projects}

\begin{longtable}{r|p{4.5in}}

    March 2014 & \textbf{Branch Predictor Simulator} \\
    Computer Systems 3 & \textbf{Github: TODO} \\
    SUNY Binghamton & Simulator for various branch predictors, including the YAGS predictor \\
    & \begin{itemize}
        \item Heaviliy utilized C++11 features such as type inference, lambdas, and variadic templates
        \item Tested using Python script
    \end{itemize} \\

    \multicolumn{2}{c}{} \\

    August 2013 & \textbf{Malloc Lab} \\
    Computer Systems 2 & \textbf{Github: \url{https://github.com/tenyoung795/malloclab}} \\
    SUNY Binghamton & Implementation of C malloc \\
    & \begin{itemize}
        \item Segregated free lists by size
        \item Doubly-linked free lists
        \item Singly-linked heap blocks
    \end{itemize}
\end{longtable}

\end{document}
